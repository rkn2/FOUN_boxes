% v1.04 - May 2023

\documentclass[]{interact}

\usepackage{epstopdf}% To incorporate .eps illustrations using PDFLaTeX, etc.
\usepackage{color} % doh

\usepackage{subfigure}% Support for small, `sub' figures and tables
%\usepackage[nolists,tablesfirst]{endfloat}% To `separate' figures and tables from text if required
\usepackage{longtable}
\usepackage{tabularx}
\usepackage{ltablex}  % This combines longtable and tabularx
\usepackage{colortbl}

\usepackage{gensymb}
\usepackage{multirow}

\usepackage{xcolor}

\usepackage{longtable}
\usepackage{tabularx}
\usepackage{booktabs}

\usepackage[hyphens]{url} % to break urls in bibliography
\usepackage[breaklinks=true]{hyperref} % LaTeX cross references become hyperlinks in pdf output


\usepackage{tikz} % Useful for drawing images, used for creating the frontpage
\usetikzlibrary{positioning} % Additional library for relative positioning 
\usetikzlibrary{calc} % Additional library for calculating within tikz

% Defines a command used by tikz to calculate some coordinates for the front-page
\makeatletter
\newcommand{\gettikzxy}[3]{%
  \tikz@scan@one@point\pgfutil@firstofone#1\relax
  \edef#2{\the\pgf@x}%
  \edef#3{\the\pgf@y}%
}
\makeatother

\usepackage{pdflscape}

\usepackage[square,sort&compress]{natbib}% Citation support using natbib.sty
\bibliographystyle{tfcad.bst} % Use the style of your choice   

\hypersetup{colorlinks=true,linkcolor=blue,urlcolor=blue}
\usepackage{soul}


\usepackage{longtable}
\usepackage{caption}
% what they recommend
\usepackage{array}
\usepackage{enumitem}
\usepackage{amssymb}
\usepackage[demo]{graphicx}
%\widowpenalty10000
%\clubpenalty10000 % to avoid sinle lines at the beginning or bottom of pages

\begin{document}

\articletype{Research Article}

\title{A Data-Driven Approach to Structural Vulnerability Assessment and Intervention Planning for Adobe Structures: A Case Study of Fort Union National Monument}

\author{
\name{Mina Savoroliya \textsuperscript{a}, Evan OJ \textsuperscript{b}, Frank Matero \textsuperscript{c}, Thomas Boothby \textsuperscript{a}, Rebecca Napolitano \textsuperscript{a}, \thanks{CONTACT Rebecca Napolitano. Email: nap@psu.edu}}
\affil{\textsuperscript{a}The Pennsylvania State University, Architectural Engineering Department, PA, USA}
\affil{\textsuperscript{b}NPS}
\affil{\textsuperscript{c}UPenn}
}

\maketitle

\begin{abstract}
This study presents a data-driven approach to assess structural vulnerability and develop intervention strategies for adobe structures, using Fort Union National Monument (FOUN) as a case study. The research integrates multiple analytical methods, including correlation analysis, Factor Analysis, and Random Forest machine learning, to identify and prioritize Necessary and Sufficient Conditions (NSCs) affecting adobe structure behavior. The study analyzed various degradation mechanisms, including adobe wall deterioration, foundation issues, and shelter coat damage. 
Data collection involved on-site surveys, archival research, and LiDAR scanning. The analysis revealed key latent factors driving degradation, specifically identifying distinct clusters of damage related to sill deterioration, surface/lintel issues, and structural instability. Random Forest analysis highlighted that geometric properties, specifically foundation height and wall height, are the most significant predictors of overall degradation, outweighing the impact of specific past treatments. Based on these findings, intervention matrices were developed, incorporating preservation standards and durability estimates. 
The results demonstrate the effectiveness of this data-driven methodology in identifying vulnerabilities and guiding preservation strategies. The study provides a structured approach to prioritizing interventions, balancing historical preservation needs with structural integrity concerns. This methodology offers a framework for the assessment and preservation of adobe structures, potentially applicable to other historical sites facing similar challenges.

\end{abstract}

\begin{keywords}
adobe structures; structural vulnerability assessment; data-driven analysis; intervention planning; structural health monitoring; factor analysis
\end{keywords}

\section{Introduction}

Adobe structures, emblematic of the American Southwest's cultural heritage, represent a unique intersection of history and engineering \cite[]{houben2004}. 
Their earthen composition ties them intrinsically to their landscapes, yet also renders them vulnerable to environmental challenges \cite[]{richards2019, reyes2019, fuentes2019}.
The study of earth-based construction materials has seen significant advancements over the past two decades. 
Early research laid the groundwork for understanding these materials' structural properties. 
\cite[]{J2001} examined the reinforcement behavior of steel bars in cement-stabilized rammed earth, providing insights into reinforcement techniques. 
Building on this, \cite[]{Vasilios2008} investigated the compressive behavior of rammed earth structural elements, offering insights into load eccentricity and slenderness effects.
As the field progressed, durability became a key focus. 
\cite[]{Bui2009} investigated the durability of rammed earth walls, particularly when reinforced with natural hydraulic lime, demonstrating the material's ability to withstand natural weathering. 
This research was complemented by \cite[]{Fratini2011}, who characterized earth-based construction materials in historical structures, identifying optimal clay content for best mechanical performance.

The early 2010s saw a shift towards more sophisticated analysis methods. 
\cite[]{Ciancio2013} proposed a new method for analyzing the structural capacity of unreinforced rammed earth walls under lateral wind forces; \cite[]{Bui2014} focused on failure mechanisms of rammed earth walls, identifying stress concentrations at loaded zones as important considerations; \cite[]{Illampas2014} integrated laboratory testing with finite element simulations for more accurate structural evaluations under horizontal loading conditions, highlighting the influence of weak bonding between masonry units and mortar joints on seismic performance.
Long-term behavior and advanced modeling techniques became prominent in subsequent years. \cite[]{Bui2015} explored the aging and creep behavior of rammed earth, contributing to our knowledge of its long-term structural integrity. 
\cite[]{Bui2016} further advanced numerical analysis techniques by utilizing the discrete element method to simulate rammed earth behavior under various loading conditions.

Recent studies have investigated more specific aspects of earth-based materials. 
\cite[]{Characterization-mechanical} offered an overview of unstabilized rammed earth, examining aspects like compressive strength and thermal performance; \cite[]{non-industrial} examined the compression behavior of rammed earth, establishing a basis for understanding its mechanical response under varying loading conditions.
Environmental factors have also been a recent focus. \cite[]{Effectofmoisture} found that slight increases in moisture content did not significantly affect wall strength, while \cite[]{Impactofrelative} explored the effects of relative humidity changes on compacted earth's mechanical properties. 
Lastly, \cite[]{Ioannou} provided a stress-strain equation for adobe bricks, essential for structural design, further refining our understanding of these materials.

Seismic behavior has been a focus for several researchers. 
\cite[]{Varum2015} conducted in-depth studies on the mechanical properties of adobe units and mortars, performing strength and joint shear tests. 
\cite[]{Sathiparan2018} examined the importance of roof and diaphragm connectivity in seismic performance, revealing the benefits of PP-band retrofitting. \cite[]{Xekalakis2023} demonstrated how wooden ring beams significantly contribute to seismic resilience in adobe structures. 
\cite[]{Rafi2022} proposed a seismic strengthening scheme using steel wire mesh and cement-sand mortar, validated through shaking table tests.
\cite[]{hart2023} experimentally subjected extant adobe block construction walls to a local 100-year return interval rainfall intensity, highlighting the importance of data-driven modeling in targeting preservation methods effectively.

Data-driven approaches, particularly Principal Component Analysis (PCA) and correlation analysis, have gained prominence in structural health monitoring, offering new avenues for preservation and risk mitigation in historic adobe structures. 
The evolution of these techniques has significantly enhanced our ability to detect and analyze structural changes while accounting for environmental factors.
Early research in this field focused on developing methods to differentiate between environmental effects and actual structural damage. 
\cite[]{Yan2005} introduced a PCA-based method for detecting structural damage that integrated environmental factors without direct measurement. 
They further refined this concept in a subsequent study \cite[]{Yan2005b}, applying local PCA in a clustered data environment to address non-linear complexities. 
This work laid the foundation for more sophisticated analysis techniques in structural health monitoring.
Building on these advancements, \cite[]{Bellino2010} demonstrated PCA's effectiveness in both Linear Time-Invariant and Linear Time-Varying systems. 
Through experiments on a clamped-free beam and numerical simulations of a train crossing a railway bridge, their work showcased PCA's ability to discern damage from environmental influences, further validating its utility in structural health monitoring.

The application of these techniques to real-world structures was exemplified by \cite[]{Magalhães}, who analyzed data from a concrete arch bridge in Porto. 
They employed static and dynamic regression models enhanced by PCA to establish relationships between natural frequencies and influencing factors, demonstrating the practical application of these methods in complex, real-world environments.
In the realm of vibration-based structural health monitoring, \cite[]{Ubertini2013} implemented a system for the San Pietro belltower in Perugia. 
Using ambient vibration tests and machine learning methods including PCA, they were able to identify post-earthquake anomalies in structural behavior. Their approach, which employed novelty analysis for damage detection, showcased the potential of these techniques in preserving historic structures.

More recently, \cite[]{Zucconi2017} expanded the application of PCA to large-scale scenario analysis. 
They developed a model to predict the habitability of unreinforced masonry buildings post-earthquake, utilizing PCA to analyze over 60,000 buildings and identify seven parameters impacting usability. 
This study demonstrated the potential of machine learning techniques for scenario analysis and planning in regions with similar building characteristics, highlighting the scalability of these methods.

As technology continues to advance, these methods are likely to play an increasingly important role in safeguarding not only civil infrastructure but also our architectural heritage.
In the realm of preservation, \cite[]{Dehghan} used PCA to assess the effectiveness of historic buildings repurposed as boutique hotels. 
Their study distilled 24 indicators into three principal components, providing actionable insights for stakeholders in making user-centric improvements.
This study demonstrates the power of data-driven techniques in enhancing our understanding and management of historic structures, offering new tools for preservation and risk mitigation.

\section{Research aim}

This study aimed to develop and validate a data-driven approach for assessing structural vulnerability and creating effective intervention plans for adobe structures, using FOUN as a case study. 
Specifically, we addressed the following research questions:
(1) How can data-driven analysis techniques, including correlation analysis, Factor Analysis, and Random Forest machine learning, be effectively applied to identify and prioritize Necessary and Sufficient Conditions (NSCs) affecting adobe structure behavior? 
(2) To what extent can these data-driven methods enhance our understanding of the complex interrelationships between various degradation mechanisms in adobe structures?
(3) How can the insights gained from data-driven analysis be translated into practical, prioritized intervention strategies that balance historical preservation needs with structural integrity concerns?
By addressing these questions, this study sought to bridge the gap between traditional preservation methods and modern data analysis techniques, offering a new framework for the assessment and preservation of adobe structures. 
The results demonstrate how data-driven approaches can lead to more informed, efficient, and effective preservation strategies.

\section{Case study}

\begin{figure}[h]
    \centering
\includegraphics[width=1\textwidth]{Images/FOUN.png}
    \caption{Fort Union National Monument \cite[]{map}}
    \label{fig:whole_site}
\end{figure}

FOUN is a historically significant site adjacent to the historic Santa Fe Trail. It was established in 1851, shortly after New Mexico became a U.S. territory. Over its history, three distinct structures were built in proximity, each serving a different strategic purpose. This study focuses on the hospital building located at the southeast end of the site, which is part of the third instance of Fort Union (Figure~\ref{fig:whole_site}). 

\begin{figure}[h]
    \centering
\includegraphics[width=0.8\textwidth]{Images/land.png}
    \caption{FOUN site plan \cite[]{map}}
    \label{fig:whole_site}
\end{figure}

The Third Fort hospital at FOUN was constructed between late 1863 and early 1864 and remained in use until 1891 when the last soldiers left for Fort Wingate and while Fort Union was vacated. 
Since its abandonment in 1891, the hospital complex (Figure~\ref{fig:fort_union_hospital_plan}), along with the rest of the fort, has been subject to various environmental and human-induced challenges. 
These include exposure to the elements, natural deterioration processes, and periods of limited maintenance before its designation as a National Monument. 
The site also experienced some salvage of materials prior to NPS management.

\begin{figure}[h!]
    \centering
    \includegraphics[width=0.8\textwidth]{Images/hospital.png}
    \caption{The plan of hospital at FOUN}
    \label{fig:fort_union_hospital_plan}
\end{figure}

Since the 1950s the National Park Service (NPS) has made continual efforts to stabilize the remnant walls, including an array of wall bracings and other supports, but certain problems with the walls require unusual efforts to maintain them in place and preserve the remaining original fabric. The hospital complex includes the hospital itself, the inner courtyard, and areas enclosed by the surrounding compound wall foundations and exterior to the hospital and inner courtyard. 

\section{Materials and methods}
\subsection{Data collection}
\subsubsection{For adobe walls}
The dataset documents adobe structural damage at FOUN. Each entry represents a wall section identified by a unique ID and includes features and scores reflecting the condition of the adobe structures. 
Data sources include photographic documentation, Lidar scans, architectural plans provided by the National Park Service, and archival research identifying past treatment efforts. 
Data was collected using a rapid assessment survey (RAS), complemented by an illustrated glossary.
Figure~\ref{fig:RASsample} shows a sample of this illustrated glossary, and additional information can be found in Section \ref{sec:ras}.

\begin{figure}[h!]
    \centering
    \includegraphics[width=1\textwidth]{Images/RASsample.png}
    \caption{Example pages from the illustrated glossary that accompanied the RAS}
    \label{fig:RASsample}
\end{figure}

The dataset includes the following features, each with an associated scoring system to quantify the level of damage or presence of specific attributes. 
Definitions about orientation can be found in Section \ref{sec:orient}. 
\begin{itemize}
    \item \textbf{Coat Cracking (Orientations 1 \& 2):}  Represents the level of cracking in the shelter coat.
        \begin{itemize}
            \item 0: No Damage
            \item 1-5: Damage Level (increasing severity)
        \end{itemize}
    \item \textbf{Coat Loss (Orientations 1 \& 2):} Represents the level of loss in the shelter coat.
        \begin{itemize}
            \item 0: No Damage
            \item 1-5: Damage Level (increasing severity)
        \end{itemize}
    \item \textbf{Structural Cracking:} Represents the presence and severity of structural cracks.
        \begin{itemize}
            \item 0: No Damage
            \item 1-5: Damage Level (increasing severity)
            \item 6: Wall Destroyed
        \end{itemize}
    \item \textbf{Cracking at Wall Junction:} Represents the presence and severity of cracks at wall junctions.
        \begin{itemize}
            \item 0: No Damage
            \item 1-5: Damage Level (increasing severity)
            \item 6: No Wall Junction
        \end{itemize}
    \item \textbf{Lintel Deterioration:} Represents the condition of the lintel (if present).
        \begin{itemize}
            \item 0: No Damage
            \item 1-5: Damage Level (increasing severity)
            \item 6: No Lintel
        \end{itemize}
    \item \textbf{Foundation Displacement (Elevations 1 \& 2):} Represents the displacement of the foundation.
        \begin{itemize}
            \item 0: No Damage
            \item 1-5: Damage Level (increasing severity)
            \item -1: Missing Data
        \end{itemize}
    \item \textbf{Foundation Mortar Condition (Elevations 1 \& 2):} Represents the condition of the mortar in the foundation.
        \begin{itemize}
            \item 0: No Damage
            \item 1-5: Damage Level (increasing severity)
            \item -1: Missing Data
        \end{itemize}
    \item \textbf{Sill (Orientations 1 \& 2):} Represents the condition of the sill (if present).
        \begin{itemize}
            \item 0: No Sill
            \item 1-5: Damage Level (increasing severity)
            \item 6: Sill Destroyed
        \end{itemize}
    \item \textbf{Cap Deterioration:} Represents the condition of the wall cap (if present).
        \begin{itemize}
            \item 0: No Cap
            \item 1-5: Damage Level (increasing severity)
            \item 6: Wall Destroyed
        \end{itemize}
    \item \textbf{Surface Loss at Top, Mid, and Low Level:} Represents the level of surface erosion at different wall levels.
        \begin{itemize}
            \item 0: No Damage
            \item 1-5: Damage Level (increasing severity)
        \end{itemize}
    \item \textbf{Out of Plane:} Represents the degree to which the wall is leaning or racked.
        \begin{itemize}
            \item 0: No Damage
            \item 1-5: Damage Level (increasing severity)
        \end{itemize}
    \item \textbf{Height:} Represents the remaining height of the wall.
        \begin{itemize}
            \item 1: No Damage (Full Height)
            \item 2-5: Damage Level (decreasing height)
        \end{itemize}
    \item \textbf{Foundation Height:}  Represents the exposed height of the foundation in inches. A blank cell indicates the foundation is not exposed.
    \item \textbf{Animal Activity:} Represents the level of animal activity.
        \begin{itemize}
            \item 0: No Damage
            \item 1-5: Damage Level (increasing severity)
        \end{itemize}
    \item \textbf{Foundation Stone Deterioration:} Represents the condition of the foundation stones.
        \begin{itemize}
            \item 0: No Damage
            \item 1-5: Damage Level (increasing severity)
            \item -1: Missing Data
        \end{itemize}
    \item \textbf{Bracing Score:} Represents the condition of any bracing present.  Calculated based on observed damage to the bracing.
        \begin{itemize}
            \item 0: No Bracing
            \item 1-5: Damage Level (decreasing condition)
        \end{itemize}
    \item \textbf{Fireplace:} Indicates the presence and proximity of a fireplace.
        \begin{itemize}
            \item 0: No Fireplace
            \item 1: Has Fireplace
            \item 2: Adjacent Fireplace
        \end{itemize}
    \item \textbf{Treatment:} Indicates whether the section has undergone treatment.
        \begin{itemize}
            \item 0: Not Treated
            \item 1: Treated
        \end{itemize}
    \item \textbf{Bracing:} Indicates the presence of bracing.
        \begin{itemize}
            \item 0: No Bracing
            \item 1: Has Bracing
        \end{itemize}
    \item \textbf{Point Cloud Deviation:} Standard deviation of the foundation's surface points relative to a reference plane. A smaller standard deviation suggests uniform, consistent behavior in foundation blocks' alignment with the reference plane. A larger standard deviation indicates more variability in alignment differences.
    \item \textbf{Point Cloud Mean:} Average vertical position of the foundation's surface relative to a reference plane. A smaller mean deviation means the foundation block closely matches the reference plane, suggesting a flat ground surface. A larger mean deviation indicates more unevenness or irregularities beneath the block.
\end{itemize}

\subsection{Analysis methods}
The Pearson Correlation was initially employed to explore the linear interrelationships among the various parameters influencing adobe structure behavior. 
By calculating Pearson correlation coefficients for each pair of parameters, we quantified the strength and direction of linear relationships. 
Coefficients approaching +1 or -1 indicated strong positive or negative linear associations, respectively, while values near 0 suggested a negligible linear relationship. 
This process yielded a correlation matrix, providing a comprehensive overview of the linear interdependencies between all parameter pairs. To ensure the robustness of our findings, we also calculated p-values for each correlation coefficient. 
A small p-value (e.g., p < 0.05) indicated strong evidence against the null hypothesis (no correlation), suggesting that the observed correlation was statistically significant and unlikely due to random chance. 
We then focused on interpreting the statistically significant correlations, moving beyond simply identifying the 'what' to understanding the 'why' behind these relationships. 
This involved analyzing the context of each correlation to discern the underlying theoretical or practical reasons driving the observed association, thereby translating statistical findings into actionable insights for decision-making.

Factor Analysis was employed to identify latent variables that explain the pattern of correlations within the observed degradation metrics. 
Unlike PCA, which focuses on explaining the total variance, Factor Analysis focuses on the common variance among variables, making it particularly suitable for identifying underlying constructs such as "Structural Integrity" or "Surface Wear" from a set of observed damage indicators. 
We used the `FactorAnalyzer` module in Python with varimax rotation to achieve a simpler and more interpretable structure. 
This method allowed us to group multiple degradation scores (e.g., cracking, loss, deterioration) into a smaller number of meaningful factors, reducing the dimensionality of the data while preserving the essential information about the structural health.

Building upon the insights gained from both correlation analysis and Factor Analysis, we further refined our understanding by employing Random Forest. 
Random Forest is a supervised learning algorithm capable of modeling complex, non-linear relationships between predictor variables and a target variable. 
In this study, we used Random Forest to predict the overall degradation score (`Total Scr`) based on geometric features (e.g., wall height, foundation height) and treatment history. 
This predictive capability allowed us to identify the most important factors contributing to degradation. 
Random Forest offers several advantages in this context: it is robust to multicollinearity, handles both continuous and categorical predictor variables, and provides measures of variable importance, allowing us to rank the factors most influential in predicting degradation.

\subsection{Data synthesis methods}
Intervention matrices graphically represent the logic of intervention decisions, reminding stakeholders to consider multiple perspectives and options\cite[]{harris2001building}. The tool avoids formulaic responses to commonly encountered deterioration mechanisms by facilitating a structured evaluation of the various options, weighing their advantages and disadvantages, and considering cost, risk, efficacy, and potential outcomes. It also illustrates the complexity of decision-making in these contexts \cite[]{harris2001building}. In these matrices, the horizontal axis represents the NSC, while the vertical axis represents various intervention approaches, such as repointing mortar joints, installing a new shelter coat, improving drainage around the foundation, or adding structural bracing. The cells within the matrix represent a qualitative assessment of the effectiveness, cost, and potential risks associated with each intervention approach for a given NSC. For example, a cell might indicate that repointing mortar joints is highly effective for addressing cracking at wall junctions but has a moderate cost and a low risk of negatively impacting the historical fabric.
The initial step in developing an intervention matrix involves determining the NSCs.
This process uses the outputs of ranked Pearson correlations, Factor Analysis, and Random Forest to identify which features show the strongest associations with others.  

\section{Results and discussion}

\subsection{Understanding damages to adobe walls}

\subsubsection{Correlation Analysis for Damage to Adobe Walls}
Pearson correlation analysis was performed to quantify the linear relationships between degradation metrics and structural features across the 67 wall sections. The analysis revealed several strong correlations with the Total Score (Total Scr), which serves as an aggregate measure of degradation severity. Cap Deterioration exhibited the strongest correlation with Total Scr (r = 0.61), followed by Out of Plane movement (r = 0.58), Height (r = 0.55), and Structural Cracking (r = 0.53). These positive correlations indicate that taller walls and those exhibiting more pronounced structural issues such as leaning or cap failure tend to accumulate higher overall degradation scores. The correlation between Bracing and Total Scr (r = 0.26) warrants particular attention. This positive relationship suggests that braced walls often exhibit higher degradation scores, which likely reflects the fact that bracing is typically installed as a reactive measure on walls that have already experienced significant deterioration rather than as a preventive intervention.

\begin{figure}[h]
    \centering
    \includegraphics[width=0.8\textwidth]{texfigures/correlation_heatmap.png}
    \caption{Correlation Matrix of Key Degradation Metrics.}
    \label{fig:correlation_heatmap}
\end{figure}

\subsubsection{Factor Analysis of Degradation Patterns}

Factor Analysis was employed to identify latent variables underlying the observed degradation patterns. This technique differs from simple correlation in that it seeks to explain the common variance among multiple observed variables through a smaller number of unobserved factors. A three-factor model with varimax rotation was selected based on eigenvalue analysis and interpretability criteria. The factor loadings are presented in Table \ref{tab:fa_loadings}.

\begin{table}[h]
\centering
\caption{Factor Analysis Loadings (Varimax Rotation)}
\label{tab:fa_loadings}
\begin{tabular}{lccc}
\toprule
\textbf{Variable} & \textbf{Factor 1} & \textbf{Factor 2} & \textbf{Factor 3} \\
\midrule
Sill 1 & \textbf{0.91} & 0.07 & 0.15 \\
Sill 2 & \textbf{0.98} & 0.12 & 0.04 \\
Coat 1 Cracking & -0.18 & \textbf{0.72} & 0.11 \\
Coat 1 Loss & 0.06 & 0.47 & 0.23 \\
Lintel Deterioration & 0.07 & \textbf{0.61} & -0.33 \\
Coat 2 Cracking & 0.06 & -0.37 & \textbf{0.58} \\
Structural Cracking & 0.08 & 0.01 & \textbf{0.50} \\
Out of Plane & -0.06 & 0.17 & \textbf{0.40} \\
\bottomrule
\end{tabular}
\end{table}

The first factor, which we term Architectural Detail Vulnerability, is heavily loaded on Sill 1 (0.91) and Sill 2 (0.98), indicating that sill degradation represents a distinct and independent failure mechanism. Sills function as water-shedding elements at the base of openings, and their deterioration, while often localized, can facilitate water ingress into the wall core. The near-unity loading values suggest that sill condition on opposite faces of a wall section are highly correlated, which is consistent with the hypothesis that sill failure is driven primarily by exposure to precipitation rather than by differential structural movement.

The second factor, Surface and Lintel Interaction, exhibits high loadings for Coat 1 Cracking (0.72) and Lintel Deterioration (0.61). This factor structure suggests a synergistic relationship between these two forms of damage. Surface coat failure around openings may expose wooden lintels to moisture infiltration, accelerating their decay. Conversely, lintel deflection or deterioration can induce tensile stresses in the surrounding plaster, leading to cracking. This finding has important implications for intervention planning, as it indicates that lintels and their surrounding surface coats should be treated as a coupled system rather than as independent elements.

The third factor, which we interpret as Global Structural Instability, groups Structural Cracking (0.50), Out of Plane movement (0.40), and Coat 2 Cracking (0.58). This clustering indicates that these forms of damage share a common underlying cause, which we attribute to deep-seated structural issues such as foundation settlement, loss of lateral restraint, or material degradation in the wall core. The moderate loading values suggest that while these damage types are related, they are not perfectly correlated, which may reflect variations in the specific failure mechanisms active in different wall sections. Interventions targeting this factor must address the underlying structural deficiencies rather than merely treating surface manifestations.

\subsubsection{Feature Importance Analysis Using Random Forest}

Random Forest regression was employed to identify the most significant predictors of overall degradation. Unlike the correlation and factor analyses, which are descriptive in nature, Random Forest provides a predictive framework that can capture nonlinear relationships and interactions between variables. The model was trained using geometric features and treatment history as predictors, with Total Scr as the target variable. Feature importance was quantified using the mean decrease in impurity metric. The results are presented in Figure \ref{fig:random_forest_new}.

\begin{figure}[h]
    \centering
    \includegraphics[width=0.8\textwidth]{texfigures/feature_importance.png}
    \caption{Random Forest Feature Importance for Predicting Total Degradation Score.}
    \label{fig:random_forest_new}
\end{figure}

Foundation Height and Wall Height emerged as the dominant predictors of degradation, each with an importance score of approximately 0.30. This finding represents a significant departure from conventional preservation thinking, which often emphasizes the role of specific treatments or material properties. Instead, the analysis indicates that geometric exposure is the primary driver of degradation at FOUN. Taller walls experience greater exposure to wind-driven rain and are subject to higher overturning moments, both of which accelerate deterioration. Similarly, exposed foundations are more vulnerable to basal erosion from splash-back and direct precipitation impact.

The relatively low importance scores for specific historical treatments (all below 0.05) suggest that the effectiveness of past interventions has been limited compared to the overwhelming influence of geometric factors. This does not necessarily indicate that the treatments were poorly executed or inappropriate, but rather that they have been insufficient to counteract the fundamental vulnerability created by the walls' height and exposure. This finding has important implications for future preservation planning. While we cannot alter the height of existing walls, we can prioritize monitoring and preventive measures for high-exposure zones. Such measures might include enhanced shelter coats with improved water resistance, installation of windbreaks to reduce wind-driven rain exposure, or improvements to site drainage to minimize splash-back and foundation saturation.

\subsection{Identifying significant conditions from results}

The development of intervention matrices requires the identification of Necessary and Sufficient Conditions (NSCs), which are conditions that are both required for and sufficient to produce the observed damage \citep{Harris2001}. Based on the Factor Analysis and Random Forest results, we identified four primary NSCs. Sill Deterioration, corresponding to Factor 1, represents a distinct vulnerability that requires specialized attention due to its independence from other failure mechanisms. Structural Instability, corresponding to Factor 3, encompasses a complex of cracking and out-of-plane movement that demands structural intervention rather than surface repair. Surface and Lintel Issues, corresponding to Factor 2, represent a coupled problem in which coat failure and lintel decay reinforce one another. Finally, High Geometric Exposure, identified through the Random Forest analysis as the dominant predictor of overall degradation, represents an underlying vulnerability that exacerbates all other forms of damage.

Table \ref{tab:significant_features} lists the features identified as significant in the analysis, detailing their inclusion or exclusion from the intervention matrix and the rationale.

\begingroup
\footnotesize
\setlength\tabcolsep{4pt}
\begin{longtable}{>{\raggedright\arraybackslash}p{2.5cm} c c c c >{\raggedright\arraybackslash}p{5cm}}
    \caption{Overview of significant features identified in analysis, their status in the intervention matrix, and justifications for exclusion.}
    \label{tab:significant_features} \\ % Add caption here
    \toprule
    \textbf{Key Features} & \textbf{Incl.} & \textbf{Corr.} & \textbf{FA} & \textbf{RF} & \textbf{Reason for Exclusion (if Not Included)} \\
    \midrule
    \endfirsthead

    \multicolumn{6}{c}{\tablename\ \thetable\ -- \textit{Continued from previous page}} \\
    \toprule
    \textbf{Key Features} & \textbf{Incl.} & \textbf{Corr.} & \textbf{FA} & \textbf{RF} & \textbf{Reason for Exclusion (if Not Included)} \\
    \midrule
    \endhead

    \midrule
    \multicolumn{6}{r}{\textit{Continued on next page}} \\
    \endfoot

    \bottomrule
    \endlastfoot

    Sill Deterioration & Yes & & \checkmark & & Distinct Factor 1 \\
    Structural Cracking & Yes & \checkmark & \checkmark & & Part of Factor 3 (Instability) \\
    Out of Plane & Yes & \checkmark & \checkmark & & Part of Factor 3 (Instability) \\
    Coat Cracking & Yes & & \checkmark & & Part of Factor 2 \\
    Lintel Deterioration & Yes & & \checkmark & & Part of Factor 2 \\
    Foundation Height & Yes & & & \checkmark & Top Predictor in RF \\
    Wall Height & No & \checkmark & & \checkmark & Associated with exposure, difficult to treat directly \\
    Bracing & Yes & \checkmark & & \checkmark & Significant predictor \\
\end{longtable}
\normalsize
\endgroup

\subsection{FOUN Intervention Matrices}

With the NSCs identified through the combined application of correlation analysis, Factor Analysis, and Random Forest regression, intervention matrices were constructed following the framework proposed by \citet{Harris2001}. The horizontal axis of each matrix represents different intervention approaches: abstention, mitigation, reconstitution, substitution, circumvention, and acceleration. The vertical axis lists the NSCs, ranked according to their importance as determined by the factor loadings and feature importance scores. Each cell in the matrix contains a numerical score representing the product of the NSC priority and the intervention approach's alignment with preservation standards. Higher scores indicate interventions that address high-priority conditions while maintaining fidelity to preservation principles.

Tables \ref{tab:intervention_matrix_adobe} and \ref{tab:intervention_matrix_foundation} present the intervention matrices for adobe wall degradation and foundation degradation, respectively. The scoring system incorporates two primary criteria. First, each NSC is assigned a priority value based on its factor loading or feature importance score, with values ranging from 0 to 10. Second, each intervention approach is assigned a preservation standards score reflecting its compatibility with the Secretary of the Interior's Standards for the Treatment of Historic Properties. Mitigation approaches, which slow deterioration while preserving original fabric, receive the highest scores (6), followed by circumvention (5), reconstitution (4), and substitution (2). Abstention and acceleration, representing no action and removal respectively, are assigned scores of 0.

\begingroup
\footnotesize
\setlength\tabcolsep{3pt}
\begin{longtable}{>{\raggedright\arraybackslash}p{2cm} p{1.8cm} p{1.8cm} p{1.5cm} p{1.8cm} p{1.5cm} p{1.5cm}}
    \caption{Intervention matrix for adobe degradation.}
    \label{tab:intervention_matrix_adobe} \\ % Add caption here
    \toprule
    \textbf{NSC} & \textbf{Abstention} & \textbf{Mitigation} & \textbf{Recon} & \textbf{Substitution} & \textbf{Circum} & \textbf{Accel} \\
    \midrule
    \endfirsthead

    \multicolumn{7}{c}{\tablename\ \thetable\ -- \textit{Continued from previous page}} \\
    \toprule
    \textbf{NSC} & \textbf{Abstention} & \textbf{Mitigation} & \textbf{Recon} & \textbf{Substitution} & \textbf{Circum} & \textbf{Accel} \\
    \midrule
    \endhead

    \midrule
    \multicolumn{7}{r}{\textit{Continued on next page}} \\
    \endfoot

    \bottomrule
    \multicolumn{7}{p{\textwidth}}{\textit{Note: NSC = Necessary and Sufficient Condition, Recon = Reconstitution, Circum = Circumvention, Accel = Acceleration}} \\
    \endlastfoot
    General Mechanism (0) & Accept the condition of and the rate of continuing expansion (0) & & & & & Demolish damaged member or members, and do not replace (0) \\
    \midrule
    Structural Instability (Factor 3) (9) & & Apply protective shelter coat (54) & Install flashing (36) & Use fiber-reinforced adobe (18) & Adjust structural elements/bracing (45) & \\
    \midrule
    Sill Deterioration (Factor 1) (8) & & Apply lime-based plaster (48) & Rebuild sill with original mix (32) & Use durable materials (16) & & \\
    \midrule
    Lintel/Surface Issues (Factor 2) (7) & & Install drip edge (42) & Replace with identical materials (28) & Replace with Cedar (14) & Support adjustments (35) & \\
    \midrule
    Out of Plane (7) & & & & Rebuild wall with improved foundation (14) & Secondary bracing system (35) & \\
\end{longtable}
\normalsize
\endgroup

The adobe degradation intervention matrix (Table \ref{tab:intervention_matrix_adobe}) reflects the factor structure identified in the statistical analysis. Structural Instability (Factor 3) receives the highest priority score (9), consistent with its role as a latent variable encompassing multiple forms of damage. The highest-scoring intervention for this NSC is the application of a protective shelter coat (score: 54), which represents a mitigation approach that can slow water infiltration without altering the wall's structural configuration. Circumvention through structural bracing receives the second-highest score (45), reflecting the need to address the underlying instability while recognizing that bracing introduces non-original elements.

Sill Deterioration (Factor 1) is assigned a priority score of 8, reflecting its distinct nature as an independent failure mechanism. The recommended mitigation approach involves applying lime-based plaster (score: 48), which is compatible with the original adobe substrate and can be renewed periodically. Reconstitution through rebuilding with the original adobe mix receives a lower score (32), as it requires removal of deteriorated material and may result in loss of original fabric.

Surface and Lintel Issues (Factor 2) are assigned a priority score of 7. The coupling between these two forms of damage, as revealed by the factor analysis, suggests that interventions must address both components simultaneously. The installation of drip edges represents a mitigation approach (score: 42) that can reduce water infiltration at the vulnerable lintel-wall interface. Circumvention through support adjustments (score: 35) may be necessary in cases where lintel deflection is contributing to coat cracking.

\begingroup
\footnotesize
\setlength\tabcolsep{3pt}
\begin{longtable}{>{\raggedright\arraybackslash}p{2cm} p{1.8cm} p{1.8cm} p{1.5cm} p{1.8cm} p{1.5cm} p{1.5cm}}
    \caption{Intervention matrix for foundation degradation.}
    \label{tab:intervention_matrix_foundation} \\ % Add caption here
    \toprule
    \textbf{NSC} & \textbf{Abstention} & \textbf{Mitigation} & \textbf{Recon} & \textbf{Substitution} & \textbf{Circum} & \textbf{Accel} \\
    \midrule
    \endfirsthead

    \multicolumn{7}{c}{\tablename\ \thetable\ -- \textit{Continued from previous page}} \\
    \toprule
    \textbf{NSC} & \textbf{Abstention} & \textbf{Mitigation} & \textbf{Recon} & \textbf{Substitution} & \textbf{Circum} & \textbf{Accel} \\
    \midrule
    \endhead

    \midrule
    \multicolumn{7}{r}{\textit{Continued on next page}} \\
    \endfoot

    \bottomrule
    \multicolumn{7}{p{\textwidth}}{\textit{Note: NSC = Necessary and Sufficient Condition, Recon = Reconstitution, Circum = Circumvention, Accel = Acceleration}} \\
    \endlastfoot

    General Mechanism (0) & Accept the condition of the section and rate of continuing expansion (0) & & & & & Demolish damaged member, and do not replace (0) \\
    \midrule
    Exposed Foundation Height (8) & & Add fill around the foundation (48) & & & Construct a barrier/retaining wall (40) & \\
    \midrule
    Foundation Stone Deterioration (7) & & Slope ground for drainage (42) & Replace stones with similar ones (28) & Replace with durable materials (14) & Bond exterior face to core (35) & \\
\end{longtable}
\normalsize
\endgroup

The foundation degradation intervention matrix (Table \ref{tab:intervention_matrix_foundation}) assigns the highest priority to Exposed Foundation Height (score: 8), consistent with the Random Forest analysis that identified this as the single most important predictor of overall degradation. The recommended mitigation approach involves adding fill around the foundation (score: 48) to reduce the exposed height and thereby minimize vulnerability to splash-back and direct precipitation impact. This intervention is reversible and does not alter the foundation itself, making it highly compatible with preservation standards. Circumvention through the construction of a barrier or retaining wall (score: 40) offers an alternative that may be appropriate in locations where adding fill would alter site drainage patterns or obscure archaeological features.

Foundation Stone Deterioration receives a priority score of 7. The recommended mitigation approach involves sloping the ground to improve drainage (score: 42), which addresses the underlying cause of deterioration rather than merely treating its symptoms. This approach is consistent with the principle of addressing NSCs rather than simply repairing damage. Circumvention through bonding the exterior face to the core (score: 35) may be necessary in cases where individual stones have become detached but retain sufficient integrity to be stabilized in place.

\section{Conclusions}

This study employed a multi-method data-driven approach to identify and prioritize Necessary and Sufficient Conditions affecting adobe structure behavior at Fort Union National Monument. The integration of correlation analysis, Factor Analysis, and Random Forest machine learning provided complementary insights that would not have been apparent through any single analytical technique. Factor Analysis revealed three distinct latent structures of degradation: Sill Deterioration, Surface and Lintel Interaction, and Global Structural Instability. This categorization has important implications for intervention planning, as it indicates that different forms of damage require fundamentally different treatment approaches. Interventions targeting Sill Deterioration can focus on localized repairs to water-shedding elements, while those addressing Global Structural Instability must address deep-seated issues such as foundation settlement or loss of lateral restraint.

The Random Forest analysis provided an important insight that challenges conventional preservation thinking. Geometric factors, specifically foundation height and wall height, emerged as the dominant predictors of overall degradation, with importance scores approximately six times higher than those of any specific historical treatment. This finding suggests that the primary driver of degradation at FOUN is not the failure of past interventions, but rather the fundamental vulnerability created by the walls' exposure to environmental forces. The implication for preservation planning is significant. While repair of existing damage remains necessary, long-term preservation of the site will require a shift in emphasis toward managing exposure through preventive measures such as enhanced shelter coats, improved drainage, and possibly the installation of windbreaks to reduce wind-driven rain.

The intervention matrices developed from these analyses provide a structured framework for prioritizing preservation actions. By combining the NSC priorities derived from statistical analysis with preservation standards scores, the matrices identify interventions that address the most significant conditions while maintaining fidelity to preservation principles. The highest-scoring interventions are generally those that employ mitigation or circumvention approaches, which slow deterioration or address underlying causes without removing original fabric. This alignment between data-driven priorities and preservation philosophy suggests that the methodology can serve as a bridge between engineering analysis and preservation practice.

The methodology demonstrated in this study is potentially applicable to other historic adobe sites facing similar challenges. The combination of rapid assessment survey data, statistical analysis, and intervention matrix development provides a replicable framework that can be adapted to different contexts. Future research might extend this approach by incorporating temporal data to track the effectiveness of interventions over time, or by developing predictive models that can forecast future deterioration under different climate scenarios. Such extensions would further enhance the utility of data-driven methods in supporting evidence-based preservation decision-making.

\section*{Acknowledgments}


\section*{Declaration of interest}

The authors report there are no competing interests to declare. 

\section*{Funding}
This material is based upon work supported by the National Science Foundation under Grant No. CMMI 2222849. Any opinions, findings, conclusions, or recommendations expressed in this material do not necessarily reflect the views of the National Science Foundation. 

\vspace{1cm}

\section*{Data availability statement}
The data that support the findings of this study, along with interactive educational notebooks demonstrating the methodology, are openly available in the GitHub repository at \url{https://github.com/rkn2/FOUN_boxes}. The repository includes synthetic datasets generated to preserve the statistical properties of the original sensitive data, as well as Jupyter notebooks for interactive diagnostics and intervention planning.

\section*{Notes on contributor(s)}
\textbf{Conceptualization} (RN, MS, EOJ, FM, TB); \textbf{Methodology} (RN, MS, EOJ, FM, TB); \textbf{Writing - Original} Draft (RN, MS, EOJ, TB); \textbf{Writing - Review \& Editing} (RN, MS, EOJ, FM, TB).


% Start the appendix
\appendix

% First appendix section
\section{Data collection}
% Content of your first appendix
\subsection{RAS} \label{sec:ras}
CAN WE ADD THE RAS IN HERE?


\subsection{Bracing details} \label{sec:bracing}

\begingroup
\footnotesize
\setlength\tabcolsep{4pt}
\begin{longtable}{>{\raggedright\arraybackslash}p{2.5cm} >{\raggedright\arraybackslash}p{2.5cm} >{\raggedright\arraybackslash}p{3cm}}
    \caption{Structural bracing attribute \cite[]{mapbracing}}
    \label{tab:bracing} \\
    \toprule
    \textbf{Bracing Number} & \textbf{Wall ID} & \textbf{Material} \\
    \midrule
    \endfirsthead

    \multicolumn{3}{c}{\tablename\ \thetable\ -- \textit{Continued from previous page}} \\
    \toprule
    \textbf{Bracing Number} & \textbf{Wall ID} & \textbf{Material} \\
    \midrule
    \endhead

    \midrule
    \multicolumn{\textbf{3}}{r}{\textit{Continued on next page}} \\
    \endfoot

    \bottomrule
    \endlastfoot
    
    B108 & 5712A-W & 1'-1/2" Angle \\
    B109 & 5712B-W & 2" Tube \\
    B110 & 5712B-W & 2" Tube \\
    B111 & 5712C-W & 2" Wood \\
    B112 & 5712C-W & 2" Wood \\
    B113 & 5712C-W & 2" Tube \\
    B114 & 5717B-W & 1'-1/2" Angle \\
    B115 & 5717B-W & 1'-1/2" Angle \\
    B116 & 5721B-N & 1'-1/2" Angle \\
    B117 & 5725A-W & 2" Tube \\
    B118 & 5723A-N & 2" Tube \\
    B119 & 5729A-E & Wood Plate \\
    B120 & 5729B-E & Wood Plate \\
    B121 & 5729B-E & Wood Plate \\
    B122 & 5705A-W & Wood Plate \\
    B123 & 5705B-N & 1'-1/2" Angle \\
    B124 & 5705E-N & 1'-1/2" Angle \\
    B125 & 5705E-N & Wood Plate \\
    B126 & 5705E-S & Wood Plate \\
    B127 & 5705E-N & Wood Plate \\
    B128 & 5705E-S & Wood Plate \\
    B129 & 5706D-E & Wood Plate \\
    B130 & 5706D-W & Wood Plate \\
    B131 & 5728B-N & 1'-1/2" Angle \\
    B132 & 5724A-S & Wood Plate \\
    B133 & 5706A-W & Wood Plate \\
    B134 & 5718B-E & 1'-1/2" Angle \\
    B135 & 5715A-N & 1'-1/2" Angle \\
    B136 & 5715A-N & 1'-1/2" Angle \\
    B137 & 5715A-N & 1'-1/2" Angle \\
\end{longtable}
\normalsize
\endgroup


\section{Data Featurization} 

\subsection{Orientation Definitions} \label{sec:orient}
\begin{figure}[htbp]
    \centering
    \caption{Orientations in the dataset}
    \label{orientations}
\begin{tikzpicture}[font=\sffamily,>=Triangle]
\scriptsize
  % Draw the building layout
  \node[rectangle, draw, minimum width=6cm, minimum height=.4cm, line width=1.5pt] (building) {};
  
  % Define the orientation arrows
  \draw[->, line width=.6pt] ($(building.north)+(0,0.2)$) -- +(0,2cm) node[above] {North};
  \draw[->, line width=.6pt] ($(building.south)-(0,0.2)$) -- +(0,-2cm) node[below] {South};
  \draw[->, line width=1pt] ($(building.east)+(0.5,0)$) -- +(2.5cm,0) node[right] {East};
  \draw[->, line width=1pt] ($(building.west)-(0.5,0)$) -- +(-2.5cm,0) node[left] {West};
  % Add labels for the building orientations with rotation for Orientation 1
  \node[align=center, above=1.1cm of building.north, rotate=90] (o1) {Orientation 1\\ Orientation 4};
  \node[align=center, below=1.1cm of building.south, rotate=90] (o2) {Orientation 2\\Orientation 3};
  \node[align=center, left=.7cm of building.west] (o3) {Orientation 2\\Orientation 3};
  \node[align=center, right=.7cm of building.east] (o4) {Orientation 1\\Orientation 4};
  
  % Draw the building's elevation lines
  \draw[dashed, line width=0.5pt] (building.north) -- (building.south);
  \draw[dashed, line width=0.5pt] (building.east) -- (building.west);
  
  % Indicate the major elevations
  \node[align=center, fill=white, inner sep=2pt] at ($(building.center)!0.5!(building.north)$) {};
  \node[align=center, fill=white, inner sep=2pt] at ($(building.center)!0.5!(building.south)$) {};
  \node[align=center, fill=white, inner sep=2pt] at ($(building.center)!0.5!(building.west)$) {};
  \node[align=center, fill=white, inner sep=2pt] at ($(building.center)!0.5!(building.east)$) {};
  
\end{tikzpicture}
\end{figure}



\begingroup
\footnotesize
\setlength\tabcolsep{4pt}
\begin{longtable}{>{\raggedright\arraybackslash}p{3cm} p{9.5cm}}
    \caption{Metric definition in the dataset}
    \label{tab:metrics} \\ % Add caption here
    \toprule
    \textbf{Metric} & \textbf{Formula} \\
    \midrule
    \endfirsthead

    \multicolumn{2}{c}{\tablename\ \thetable\ -- \textit{Continued from previous page}} \\
    \toprule
    \textbf{Metric} & \textbf{Formula} \\
    \midrule
    \endhead

    \midrule
    \multicolumn{2}{r}{\textit{Continued on next page}} \\
    \endfoot

    \bottomrule
    \endlastfoot

    Coat Weight Score & Coat Loss Score + Coat Cracking Score \\
    \midrule
    Elevation Normalized Score & (Surface Loss at Low level + Surface Loss at Mid level + Surface Loss at Top level + Sill Damage Score) for each section in the specific orientation / Maximum Damage(Surface Loss at Low level + Surface Loss at Mid level + Surface Loss at Top level + Sill Damage Score) \\
    \midrule
    Elevation Weight Score & Elevation Norm Score $\times 1000$ \\
    \midrule
    Wall Normalize Score & (Cap Deterioration + Structural Cracking + Cracking at Wall Junction + Out of Plane + Lintel Deterioration + Height) for each section / Maximum Damage(Cap Deterioration + Structural Cracking + Cracking at Wall Junction + Out of Plane + Lintel Deterioration + Height) \\
    \midrule
    Wall Weight Score & Wall Normalize Score $\times 1000$ \\
    \midrule
    Total Score & Wall Weight Score + Coat1 Weight Score + Coat2 Weight Score + Coat3 Weight Score + Coat4 Weight Score + Elevation1 Weight Score + Elevation2 Weight Score \\
    \midrule
    Wall Rank & Wall Weight Score = Higher Wall Rank \\
\end{longtable}
\normalsize
\endgroup

\bibliography{biblio.bib} 


\end{document}
